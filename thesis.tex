\documentclass[12pt,a4paper]{article}

\usepackage{fontspec}
\usepackage{polyglossia}
\usepackage[left=3cm,top=2cm,right=1cm,bottom=2cm,nohead]{geometry}
\usepackage{setspace}
\usepackage{listings}
\usepackage{color}
\usepackage{float}
\usepackage{courier}
\usepackage{bold-extra}
\usepackage{fix-cm}
\usepackage{alltt}
\usepackage{indentfirst}
\usepackage{amsmath, amsthm, amssymb}
\usepackage{url}

\defaultfontfeatures{Mapping=tex-text}

\setmainfont
    [ Path           = fonts/ ,
      UprightFont    = *-Regular,
      BoldFont       = *-Bold ,
      ItalicFont     = *-Italic ,
      BoldItalicFont = *-BoldItalic]
    {LiberationSerif}
\setsansfont
    [ Path           = fonts/ ,
      UprightFont    = *-Regular,
      BoldFont       = *-Bold ,
      ItalicFont     = *-Italic ,
      BoldItalicFont = *-BoldItalic]
    {LiberationSans}
\setmonofont
    [ Path           = fonts/ ,
      UprightFont    = *-Regular,
      BoldFont       = *-Bold ,
      ItalicFont     = *-Italic ,
      BoldItalicFont = *-BoldItalic]
    {LiberationMono}

\setmainlanguage{ukrainian}
\setotherlanguage{english}

\setstretch{1.1}

\begin{document}
\pretolerance=-1
\tolerance=2300

\pagenumbering{arabic}
\pagestyle{empty}
\setlength{\parindent}{1.5cm}
\fontsize{14pt}{6mm}\selectfont

\begin{center}
  Міністерство освіти і науки, молоді та спорту України
  
  Львівський національний університет імені Івана Франка

  Факультет прикладної математики та інформатики
\end{center}

\vspace{1cm}

\begin{flushright}
  Кафедра програмування
\end{flushright}

\vspace{4cm}

\begin{center}
  {\bfseries\Large Архітектурні особливості проекту створення віртуальної електронної лабораторії BUMMEL}
\end{center}

\vspace{2cm}

\begin{small}
\begin{flushleft}\leftskip8.5cm
  Магістерська робота студента групи ПМІ-51м\\
  Михалевича І.А.\linebreak
  
  Науковий керівник:\\
  доц. Рикалюк Р.Є.
\end{flushleft}
\end{small}

\vspace{4cm}

\begin{center}
  Львів - 2013
\end{center}

\clearpage



\setstretch{1.5}
\fontsize{14pt}{6mm}\selectfont

\tableofcontents
\clearpage
\pagestyle{plain}
\section{Вступ}

1. Початок BUMMEL: коли, ідея, з чого \cite{alias}\\
2. Розробка і розвиток:\\
  2.1. зміни в ході розвитку (зростання функціоналу)\\
  2.2. людський фактор (приріст кількості учасників)\\
3. Подальший розвиток, як постановка задачі (вдосконалення програми та гармонізація потоку праці)\\
//@TODO: узгодити назви цих пунктів з розділами праці
  3.1. архітектура моделювання\\
  3.2. архітектура програми

\clearpage

\section{Моделювання логічних схем}

1. Огляд методів логічного моделювання та їх класифікації\\
  1.1. Рівні моделювання\\
  1.2. Загальний огляд логічного моделювання\\
  1.3. Процес моделювання логічних схем\\
  1.4. ... {з конспектів INTUIT}\\
2. Наявний в BUMMEL метод моделювання\\
  2.1. Класифікація\\
  2.2. Модель\cite{v1-model(element, circuit)}\\
  2.3. Алгоритм\\
3. Зміна алгоритму\\
  3.1. Затримки\\
  3.2. Черга майбутніх подій (ЧМП)\\
  3.2. Алгоритм
    3.2.1. Робота з пам’яттю, коли які зміни і т.д. („інфраструктура“ для зручного управління)

\clearpage

\section{Взаємодія з користувачем}

1. Властивості елементів (панель налаштувань)\\
2. Взаємодія з елементами/зі схемою (вікна налаштувань, натискання)

\clearpage

\section{Супутні архітектурні рішення в ході розробки}

1. Елементи меню для створення редакторів різних схем

\clearpage

\section{Отримані результати}

Внаслідок проведених досліджень моделювання було створено та окреслено ...\\
Внаслідок впровадження нового алгоритму було отримано ...\cite{web}

\clearpage

\section{Висновки}

В результаті проробленої роботи було досягнуто ...\cite{web}

\clearpage

\addcontentsline{toc}{section}{Література}
\begin{thebibliography}{9}

  \bibitem{alias}<АВТОР> \emph{<КНИГА>},
    <ВИДАВНИЦТВО> <РІК>, <К-КІСТЬ СТОРІНОК> ст.
    
  \bibitem{web}<АВТОР> \emph{<НАЗВА>} [Електронний ресурс],
    <РІК>. Режим доступу:
    \url{https://github.com/Uko/thesis-template}

\end{thebibliography}

\end{document}
