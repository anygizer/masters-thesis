\documentclass[12pt,a4paper]{article}

\usepackage{fontspec}
\usepackage{polyglossia}
\usepackage[left=3cm,top=2cm,right=1cm,bottom=2cm,nohead]{geometry}
\usepackage{setspace}
\usepackage{listings}
\usepackage{color}
\usepackage{float}
\usepackage{courier}
\usepackage{bold-extra}
\usepackage{fix-cm}
\usepackage{alltt}
\usepackage{indentfirst}
\usepackage{amsmath, amsthm, amssymb}
\usepackage{url}

\defaultfontfeatures{Mapping=tex-text}

\setmainfont
    [ Path           = fonts/ ,
      UprightFont    = *-Regular,
      BoldFont       = *-Bold ,
      ItalicFont     = *-Italic ,
      BoldItalicFont = *-BoldItalic]
    {LiberationSerif}
\setsansfont
    [ Path           = fonts/ ,
      UprightFont    = *-Regular,
      BoldFont       = *-Bold ,
      ItalicFont     = *-Italic ,
      BoldItalicFont = *-BoldItalic]
    {LiberationSans}
\setmonofont
    [ Path           = fonts/ ,
      UprightFont    = *-Regular,
      BoldFont       = *-Bold ,
      ItalicFont     = *-Italic ,
      BoldItalicFont = *-BoldItalic]
    {LiberationMono}

\setmainlanguage{ukrainian}
\setotherlanguage{english}

\setstretch{1.1}

\begin{document}
\pretolerance=-1
\tolerance=2300

\pagenumbering{arabic}
\pagestyle{empty}
\setlength{\parindent}{1.5cm}
\fontsize{14pt}{6mm}\selectfont

\begin{flushright}
  {\bfseries\small Форма № Н-9.02}
\end{flushright}

\begin{center}
  \begin{spacing}{2}
    \uppercase{
      Львівський національний університет імені Івана Франка

      Факультет прикладної математики та інформатики

      Кафедра програмування
    }
  \end{spacing}

  \vspace{6cm}


    {\bfseries\Large Магістерська робота}

    {\small (освітньо--кваліфікаційний рівень магістр)}

    \bigskip

    на тему: {\bfseries\large „Архітектурні особливості проекту створення електронної лабораторії BUMMEL“}

\end{center}

\vspace{2cm}

\begin{small}
\begin{flushleft}\leftskip8.5cm
  Виконав студент 5 курсу, групи ПМІ-51м спеціальності 8.04030201 Інформатика\\
  Михалевич І.А.\linebreak

  Керівник: доц. Рикалюк Р.Є.\linebreak

  Рецензент: \underline{\hspace{3cm}}

\end{flushleft}
\end{small}

\vspace{5cm}

\begin{center}
  Львів - 2013 року
\end{center}

\clearpage



\setstretch{1.5}
\fontsize{14pt}{6mm}\selectfont

\tableofcontents
\clearpage
\pagestyle{plain}
\section{Вступ}

Електронна лабораторія BUMMEL - це програма для моделювання логічних схем. Розробка її почалась два роки тому (2011 рік) як груповий курсовий проект. Метою даного проекту є вивчення сучасних технологій розробки програмного забезпечення та створення навчальної програми, яка може бути використана в курсі „Архітектура ЕОМ“ і буде зручнішою і коректнішою в роботі ніж ПЗ, яке використовується на даний момент.

Крім того, програма розробляється як вільний продукт \cite{thesis-foss2013} і будь-хто може долучитись до цього процесу.

Перед початком розробки було обрано, обдумано і складено архітектуру програми та алгоритм моделювання об’єктів дослідження.

За час розробки програми змінювалось практично все. Було опробувано різні підходи до організації !!!коду!!! (робочого потоку)
%\cite{тези на конферецію}
та вибрано найбільш зручні. Також з плином еволюції програми змінювались і технології з допомогою яких реалізовувався той чи інший функціонал (наприклад, SVG графіка та інструменти побудови виконавчого коду). При переході до версії 0.2 було виправлено та вдосконалено моделювання схем. Елементна база не стояла на місці і теж розширилась -- починалось все з базових елементів логіки, а зараз вже маємо елементи з пам’яттю.

Крім самого проекту змінилась, а точніше збільшилась і група людей, що над ним працює. Власне, нові люди допомагали з розширенням функціоналу та опробуванням нових технологій.
% \cite{на дипломні Романа і Юрка}

Як видно з короткого опису, BUMMEL - живий проект, який видозмінюється з усіх сторін. Але, якщо дивитись уважно, то видно, що змінювалось практично все, крім архітектури програми. Як на самому початку було обрано базувати програму на NetBeans Platform, а архітектурно на MVC, де, в силу архітектури основи, другий компонент абревіатури - V - View (NB Visual Library) також структурно являє собою MVC патерн. Тобто BUMMEL -- MMVCC NetBeans Platform програма.

Такий довгий період без втручань до цієї частини електронної лабораторії вказує на одне з двох: або архітектура була дуже добре складена і не потребує втручань, або, навпаки, вона дуже складна і всі бояться її модифікувати „раз і так працює“.

Насправді ситуація що склалась на сьогодні поєднує ці два варіанти. Ідейно архітектура підібрана добре, але в силу величини і, можна сказати, складності мови Java та каркасу NetBeans Platform, на практиці не доведена до бажаного стану. Декількома надлишковими зв’язками порушено структуру M(V=MVC)C. До цього також привів початковий задум - зробити найнижчу логіку програми універсальною, для можливості моделювати не лише логічні схеми, а й багато інших процесів, що можна відобразити через мережі, дерева та ін.

До Архітектурної проблеми додається ще функціонально-архітектурна проблема. Як виявилось, з наявним методом моделювання ми не можемо просунутись далі ніж прості схеми з елементами пам’яті. Наприклад, вже циклічні схеми з тригерами можуть дати некоректні результати.

Для того щоб програма могла і далі розвиватись в своєму основному напрямку, а не по супутніх та додаткових гілках, потрібно викристалізувати архітектуру програми та скласти чи підібрати і адаптувати алгоритм моделювання схем. Крім того, так як програма являється і дослідницькою роботою, то потрібно створити інструменти які б спрощували, власне, дослідження тих чи інших алгоритмів та підходів в даній ситуації (для поставлених цілей і з утвореними задумами).

Це все дасть можливість:
\begin{enumerate}
  \item швидше орієнтуватись в програмі та обирати область для праці (внаслідок прозорості та доступності архітектури)
  \item збільшити продуктивність команди розробників (внаслідок ширшого спектру можливостей для розвитку)
  \item в разі успіху з укладанням алгоритму моделювання логічних елементів та схем, впровадити програму в університетський навчальний курс
  \item використовувати програму як бібліотечну основу для інших задач та моделювання інших процесів
  \item спростити та вгармонізувати процес розростання команди розробників, яка зростала протягом минулих років кожного року
\end{enumerate}

Дана проблема, являє собою, так зване, вузьке місце, в графі розвитку програми.

Мета даної магістерської роботи - дослідити наявну архітектуру програми та метод моделювання, надати інструменти для подальших досліжень та експериментів з ними, вдосконалити архітектуру програми - виправити описану вище проблему. Також, у випадку успіху, зробити продукт який буде прийнятним для впровадження в навчальний курс „Архітектури ЕОМ“.

\clearpage

\section{Постановка задачі}

Вдосконалення програми та гармонізація потоку праці\\
  1. архітектура моделювання\\
  2. архітектура програми
%?TODO: узгодити назви цих пунктів з розділами праці

\clearpage

\section{Моделювання логічних схем}

1. Огляд методів логічного моделювання та їх класифікації\\
  1.1. Рівні моделювання\\
  1.2. Загальний огляд логічного моделювання\\
  1.3. Процес моделювання логічних схем\\
  1.4. ... {з конспектів INTUIT}\\
2. Наявний в BUMMEL метод моделювання\\
  2.1. Класифікація\\
  2.2. Модель\\
%\cite{v1-model(element, circuit)}
  2.3. Алгоритм\\
3. Зміна алгоритму\\
  3.1. Затримки\\
  3.2. Черга майбутніх подій (ЧМП)\\
  3.2. Алгоритм
    3.2.1. Робота з пам’яттю, коли які зміни і т.д. („інфраструктура“ для зручного управління)

\clearpage

\section{Взаємодія з користувачем}

1. Властивості елементів (панель налаштувань)\\
2. Взаємодія з елементами/зі схемою (вікна налаштувань, натискання)

\clearpage

\section{Супутні архітектурні рішення в ході розробки}

1. Елементи меню для створення редакторів різних схем

\clearpage

\section{Отримані результати}

Внаслідок проведених досліджень моделювання було створено та окреслено ...\\
Внаслідок впровадження нового алгоритму було отримано ...

\clearpage

\section{Охорона праці та безпеки в надзвичайних ситуаціях}

\subsection{Вступ}
З розвитком науково-технічного прогресу важливу роль грає можливість безпечного виконання людьми своїх трудових обов'язків.
Охорона здоров'я людей що працюють, забезпечення безпеки умов праці, ліквідація професійних захворювань і виробничого травматизму становить основу людської безпеки. Проте не варто забувати, що небезпечні чинники можуть діяти на людський організм не-поодинці, а взаємопов’язано. Але нажаль ми не можемо проаналізувати всі можливі комбінації сукупної дії небезпечних та шкідливих чинників. Невід’ємним елементом сьогодення є те, що людина проводить половину свого дня у приміщені за роботою.
Мета роботи – вивчення безпеки праці на робочому місці, вплив шкідливих чинників на працюючу особу й відчуття міри захисту від неї.
У зв’язку з цим була створена та розвивається наука про безпеку праці та життєдіяльності людини. Це комплекс заходів, вкладених у забезпечення безпеки людини у середовище проживання, збереження здоров'я, розробку методів і засобів захисту за методом зниження впливу шкідливих і найнебезпечніших чинників до допустимих значень, вироблення заходів для обмеження шкоди та ліквідацію наслідків надзвичайних ситуацій мирного й військової часу.
Під час виконання бакалаврської роботи було використано ряд різноманітного приладдя яке дає змогу опрацьовувати низку матеріалу, а саме комп’ютери, принтери, сканери. Не враховуючи приміщення в якому відбувається цей тривалий процес, а саме лабораторії, комп’ютерні класи
За умов роботи з ПК виникають наступні небезпечні та шкідливі чинники: несприятливі мікрокліматичні умови, освітлення, електромагнітні випромінювання, забруднення повітря шкідливими речовинами (джерелом яких може бути принтер, сканер), шум, вібрація, електричний струм, електростатичне поле, напруженість трудового процесу .


\subsection{Аналіз стану умов праці}
\subsubsection{Характеристика виробничого середовища та чинників трудового процесу}
\begin{itemize}
\item Магістерська робота виконувалась і оформлювалась у студентській лабораторії, яка знаходиться у приміщенні ЛНУ ім. І. Франка на філософському факультеті кафедрі психології 1А. Вул. Дорошенка 41.
\item Приміщення планово-економічного відділу розташовано на третьому поверсі п'ятиповерхового будинку. У приміщенні розташовано 3 робочих місць з комп’ютерами. Розміри даного приміщень складають: довжина – 10 м, ширина – 6 м, висота – 3,5 м, тобто загальна фактична площа складає 60 м2.
\item Необхідна площа на 3 робочих місця із установленими ПК складає 18 м2, що не перевищує фактичну. Обсяг кабінету на одного працюючого складає 70.м3, отже відповідає нормі (ДНАОП 0.00-1.31-99) [3] – не менше 20 м3. Площа одного робочого місця з відео дисплейним терміналом повинна бути не менше 6 м. кв, а об’єм не менше 20м.кб. Робочі місця розташовані від стіни з вікнами 1.5 м. а від бокових стін 1 м. Відстані між боковими поверхнями відео дисплеїв 1.2 м, а від тильного боку одного до екрану другого 2.5 м( я не знаю точно як порахувати на 10 робочих мість)(в цьому розділі  не наводьте нормативів, лише  те що реально  є. Це можна видалити)
\item Загальна кількість робочих місць 10.
\item Приміщення розташоване з північно-західною орієнтацією вікон. Мікроклімат у приміщенні забезпечує комфортне самопочуття людини. Оптимальна температура в приміщені становить: в теплий період року 23–25 °С, у холодний 22–24 °С. Відносна вологість повітря коливається в межах 60-40 \%. Швидкість руху повітря не перевищує 0.2 м/с – у теплий період часу, а у холодний 0.1 м/с. Приміщення з відеодисплейним терміналом оснащене припливно-витяжною вентиляцією та забезпечене природним вентилюванням.
\item Освітлення приміщення змішане: природне і штучне. При цьому коефіцієнт природного освітлення не менше 1,5 \%, освітленість при штучному освітлені в площині робочої поверхні становить 300-500лК. Також допускається локальне освітлення з освітленістю екрана не більше 300 лК.
\item У даному приміщені не має наявних хімічних речовин. Основним джерелом небезпек під час роботи з персональним комп’ютером є відеодисплейний термінал на основі електронно-променевої трубки. Більш безпечні відеодисплейні термінали з плазмовими та рідинно-кристалічними екранами. Крім безпосереднього впливу на організм людини, робота відеодисплейних терміналів має ще й опосередкований вплив, а саме: зумовлює порушення балансу аероіонів у зоні дихання користувача і призводить до зменшення кількості від’ємних аероіонів, які мають суттєвий вплив на імунну систему організму. Пониження імунітету людини внаслідок зменшення негативних аероіонів зумовлює інші порушення в організмі, зокрема й тих, що непов’язані з роботою на персональному комп’ютері.
\item У приміщені наявна аптечка. 
\item До засобів гасіння пожеж належать встановлені пожежні стволи, внутрішні пожежні водопроводи, вогнегасники, сухий пісок, азбестові ковдри. Пожежні крани встановлені в коридорах, на сходових клітках та вході. Приміщення забезпечені вогнегасниками. Для виявлення стадії загоряння та оповіщення використовують системи автоматичної пожежної сигналізації.
\item На робочому місці викладачі проводять повний інструктаж з  охорони праці, повідомлять де розміщені плани будівлі, додаткові входи і виходи, розміщення вогнегасників. Також повідомлять про ряд небезпечних речовин та чинників, що впливають на наш організм.
\end{itemize}
\subsubsection{Опис трудового процесу} 
Негативний вплив на організм людини виникає через неадекватне (надто велике або надто мале) навантаження на окремі системи організму. Такі перекоси у напруженні різних систем організму, що трапляються підчас роботи з відеодисплейним терміналом, зокрема, значна напруженість зорового аналізатора і довготривале малорухоме положення перед екраном, не тільки не зменшують загального напруження, а навпаки, призводять до його посилення і прояву стресових реакцій.
Виконання бакалаврської роботи належить до легких робіт згідно класифікації робіт за ступенем важкості.
Основним робочим становищем є положення сидячи. Робоча поза сидячи викликає мінімальне стомлення людини. Раціональне планування робочого місця передбачає чіткий лад і сталість розміщення предметів. Те, що потрібно виконувати частіше, лежить у зоні легкої досяжності робочого простору. 
Можна вважати, що робоче місце досить добре пристосоване для ефективного виконання поставлених завдань і не приводить до погіршення продуктивності праці та погіршення самопочуття чи здоров’я.
\subsubsection{Аналіз методів дослідження, обладнання та характеристика речовин}
Під час виконання бакалаврської роботи. використовують персональні комп’ютери та периферійні пристрої (лазерні та струменеві друки, копіювальну техніку, сканери). Негативний вплив цих пристроїв на організм людини виникає через неадекватне (надто велике або надто мале) навантаження на окремі системи організму. Такі перекоси у напруженні різних систем організму, що трапляються під час роботи з ПК, зокрема, значна напруженість зорового аналізатора і довготривале малорухоме положення перед екраном, не тільки не зменшують загального напруження, а навпаки, призводять до його посилення і прояву стресових реакцій. Найбільшому ризику виникнення різноманітних порушень піддаються: органи зору, м’язово-скелетна система, нервово-психічна діяльність, репродуктивна функція у жінок.
Робота з комп’ютером характеризується значною розумовою напругою і нервово-емоційним навантаженням, високою напруженістю зорової праці та досить великим навантаженням на м’язи рук під час роботи з клавіатурою.
У процесі роботи з комп'ютером необхідно дотримуватися правильного режиму праці та відпочинку. Інакше у людини відзначаються значна емоційна напруга зорового апарату, що може призвести до появи головного білю, дратівливості, порушення сну, почуття виснаження. Раціональний режим праці та відпочинку передбачає запровадження регламентованих перерв, рівномірний розподіл навантаження протягом робочого дня, регулярні комплекси вправ для очей, рук, хребта для  поліпшення мозкового кругообігу та психофізіологічного розвантаження.
З метою запобігання перевантаження організму як в цілому, так і окремих його функціональних систем, передусім зорового та рухового аналізаторів, центральної нервової системи, загальний час щоденної роботи з відеодисплейним терміналом треба обмежити чотирма годинами та обов’язково дотримуватись регламентованих перерв.
Робота з ПК супроводжується також виділенням значної кількості тепла, шуму, вібрації та електромагнітних випромінювань.

\subsection{Організаційно-технічні заходи}
\subsubsection{Організація робочого місця і роботи}
Санітарно-гігієнічні та ергономічні вимоги до параметрів робочого місця, розміщення обладнання, пристроїв та персонального комп’ютера на ньому, психофізіологічні особливості праці (напруженість трудового процесу).
Робоче місце – це зона трудових дій працівника, обладнана для виконання певних операцій виробничого процесу, де взаємодіють три головні елементи праці – предмет, засоби і суб’єкт праці. На одному робочому місці можуть працювати два або кілька працівників, які виконують спільне завдання.  Наукова організація робочого місця передбачає створення працівникові всіх необхідних умов для високопродуктивної і високоякісної праці за можливо менших фізичних зусиль і мінімальному нервовому напруженні та передбачає:
\begin{itemize}
\item оснащеність робочого місця відповідним основним і допоміжним устаткуванням, технологічною і організаційною оснасткою;
\item раціональне планування, тобто найзручніше і найефективніше розміщення усіх елементів робочого місця для трудового процесу;
\item створення безпечних і здорових умов праці.
Просторова організація робочого місця повинна забезпечувати:
\item відповідність планування робочого місця санітарним і протипожежним нормам і вимогам;
\item безпеку працівникам;
\item відповідність просторових відношень між елементами робочого місця, антропометричними, біомеханічними, фізіологічними, психофізіологічними і психічними можливостями людини, що працює;
\item можливість виконання основних і допоміжних операцій в робочому положенні, що відповідає специфіці трудового процесу, в раціональній робочій позі і з використанням найбільш ефективних прийомів праці;
\item вільне переміщення працівника за оптимальними траєкторіями;
\item достатню площу для розміщення обладнання, інструменту, засобів контролю, деталей та ін.
Просторові та розмірні співвідношення між елементами робочого місця повинні дозволяти:
\item розміщення працівника з врахуванням робочих рухів і переміщень згідно з технологічним процесом;
\item оптимальний огляд джерела візуальної інформації;
\item зміну робочої пози і положення;
\item раціональне розміщення основних і допоміжних засобів праці.
\end{itemize}
Обов’язковою умовою є те, що на робочому місці повинні знаходитись лише ті технічні засоби, які необхідні для виконання робочого завдання, і розміщуватися вони повинні в межах досяжності з метою виключення частих нахилів і поворотів корпусу людини, що працює.
Під час роботи з персональним комп’ютером повинні бути дотримані наступні вимоги.
Вимоги до приміщення. Площу приміщень, в яких розташовують персональні комп’ютери, визначають згідно з чинними нормативними документами з розрахунку на одне робоче місце, обладнане ПК:
\begin{itemize}
\item площа — не менше 6,0 м2;
\item об’єм — не менше 20,0 м3, з урахуванням максимальної кількості осіб, які одночасно працюють у зміні;
\item робочі місця повинні бути розташовані на відстані не менше ніж 1 м від стіни з вікном;
\item відстань між бічними поверхнями комп’ютерів має бути не меншою за 1,2 м;
\item відстань між тильною поверхнею одного комп’ютера та екраном іншого не повинна бути меншою 2,5 м;
\end{itemize}
Прохід між рядами робочих місць має бути не меншим 1 м.
Користування ПК є основним видом діяльності, то ПК і його периферійні пристрої (принтер, сканер) розміщується на основному робочому столі, як правило, з лівого боку.
Вимоги до організації робочого місця з ПК. Конструкція робочого місця користувача ПК має забезпечувати підтримання оптимальної робочої пози з такими ергономічними характеристиками: 
\begin{itemize}
\item ступні ніг — на підлозі або на підставці для ніг; 
\item стегна — в горизонтальній площині; 
\item передпліччя — вертикально; 
\item лікті — під кутом 70–90° до вертикальної площини; 
\item зап’ястя зігнуті під кутом не більше 20° відносно горизонтальної площини; 
\item нахил голови — 15–20° відносно вертикальної площини.
\end{itemize}
Якщо користування ПК є основним видом діяльності, то ПК і його периферійні пристрої (принтер, сканер) розміщується на основному робочому столі, як правило, з лівого боку. Якщо використання ПК є періодичним, то він, як правило, розміщується на приставному столі, переважно з лівого боку від основного робочого столу.
Кут між поздовжніми осями основного та приставного столів має бути 90–140°.
Висота робочої поверхні столу для ПК має бути в межах 680–800 мм, а ширина — забезпечувати можливість виконання операцій в зоні досяжності моторного поля.
Рекомендовані розміри столу: висота 725 мм, ширина 600–1400 мм, глибина 800–1000 мм.
Робочий стіл для ПК повинен мати простір для ніг висотою не менше 600 мм, шириною не менше 500 мм, глибиною на рівні колін не менше 450 мм, на рівні витягнутої ноги — не менше 650 мм.
Робочий стіл для ПК, як правило, має бути обладнаним підставкою для ніг шириною не менше 300 мм та глибиною не менше 400 мм, з можливістю регулювання по висоті в межах 150 мм та кута нахилу опорної поверхні - в межах 20°. Підставка повинна мати рифлену поверхню та бортик на передньому краї заввишки 10 мм. Застосування підставки для ніг тими, у кого ноги не дістають до підлоги, є обов’язковим.
Робоче сидіння (сидіння, стілець, крісло) користувача ПК повинно мати такі основні елементи: сидіння, спинку, стаціонарні або знімні підлокітники. У конструкцію сидіння можуть бути введені додаткові елементи, що не є обов’язковими: підголовник та підставка для ніг. 
Робоче сидіння користувача ПК повинно бути підйомно-поворотним, таким, що регулюється за висотою, кутом нахилу сидіння та спинки, за відстанню спинки до переднього краю сидіння, висотою підлокітників. Регулювання кожного параметра має бути незалежним, плавним або ступінчатим, мати надійну фіксацію.
Хід ступінчатого регулювання елементів сидіння має становити для лінійних розмірів 15–20 мм, для кутових – 2–5°. Зусилля під час регулювання не повинні перевищувати 20 Н. Ширина та глибина сидіння повинні бути не меншими за 400 мм. Висота поверхні сидіння має регулюватися в межах 400–500 мм, а кут нахилу поверхні - від 15° вперед до 5° назад. Поверхня сидіння має бути плоскою, передній край - заокругленим. Висота спинки сидіння має становити 300±20 мм, ширина - не менше 380 мм, радіус кривизни в горизонтальній площині - 400 мм. Кут нахилу спинки повинен регулюватися в межах 0–30° відносно вертикального положення. Відстань від спинки до переднього краю сидіння повинна регулюватись у межах 260–400 мм.
Для зниження статичного напруження м’язів рук необхідно застосовувати стаціонарні або знімні підлокітники довжиною не менше 250 мм, шириною 50–70 мм, що регулюються по висоті над сидінням у межах 230±30 мм та по відстані між підлокітниками в межах 350–500 мм.
Поверхня сидіння, спинки та підлокітників має бути напівм’якою, з неслизьким, ненаелектризовувальним, повітронепроникним покриттям та забезпечувати можливість чищення від бруду.
Монітор та клавіатура мають розташовуватися на оптимальній відстані від очей користувача, але не ближче 600 мм, з урахуванням розміру алфавітно-цифрових знаків та символів.
Розташування монітору має забезпечувати зручність зорового спостереження у вертикальній площині під кутом ±30° від лінії зору працівника.
Клавіатуру слід розміщувати на поверхні столу або на спеціальній, регульованій за висотою, робочій поверхні окремо від столу на відстані 100–300 мм від краю, ближчого до працівника. Кут нахилу клавіатури має бути в межах 5–15°.
Розміщення принтера або іншого пристрою введення-виведення інформації на робочому місці має забезпечувати добру видимість монітору, зручність ручного керування пристроєм введення-виведення інформації в зоні досяжності моторного поля: по висоті 900–1300 мм, по глибині 400–500 мм.
При потребі високої концентрації уваги під час виконання робіт з високим рівнем напруженості суміжні робочі місця з ПК необхідно відділяти одне від одного перегородками висотою 1,5–2 м.
Режим праці та відпочинку користувачів ПК встановлюють з урахуванням психофізіологічної напруженості їхньої праці, динаміки функціонального стану систем організму та працездатності. Раціональний режим праці та відпочинку передбачає запровадження регламентованих перерв, рівномірний розподіл навантажень протягом робочого дня, регулярні комплекси вправ для очей, рук, хребта, поліпшення мозкового кругообігу та психофізіологічне розвантаження.
З метою запобігання перевантаження організму як в цілому, так і окремих його функціональних систем, передусім зорового та рухового аналізаторів, центральної нервової системи, загальний час щоденної роботи з ПК обмежують. Робота з персональним комп’ютером вважається основною, якщо вона займає не менше як 50 % часу робочого дня чи робочої зміни.
З урахуванням характеру трудової діяльності, напруженості та важкості праці з використанням ПК під час основної роботи за восьмигодинної робочої зміни встановлюють додаткові регламентовані перерви:
\begin{itemize}
\item для розробників програм тривалістю 15 хв через кожну годину роботи;
\item для операторів персональних комп’ютерів тривалістю 15 хв через дві години роботи;
\item для операторів комп’ютерного набору тривалістю 10 хв через кожну годину роботи.
\end{itemize}
За жодних умов безперервна робота з ПК не повинна перевищувати чотири години.
За дванадцятигодинної робочої зміни протягом перших восьми годин регламентовані перерви встановлюють аналогічно до восьмигодинної робочої зміни, а протягом останніх чотирьох годин тривалістю 15 хв через кожну годину незалежно від характеру трудової діяльності.

Навчання та інструктажі з безпеки праці. Перед допуском до самостійної роботи кожен працівник має право на навчання з питань охорони праці і роботодавець зобов’язаний провести таке навчання у вигляді двох інструктажів з питань охорони праці:
\begin{itemize}
\item вступного, який проводять працівники служби охорони праці об’єкта господарювання з усіма працівниками, яких приймають на роботу незалежно від їхньої освіти та стажу роботи за програмою, в якій подають загальні питання охорони праці із врахуванням її особливостей на об’єкті господарювання;
\item первинного, який проводять керівники структурних підрозділів на робочому місці з кожним працівником до початку їхньої роботи на цьому робочому місці.
\end{itemize}
Проходження цих інструктажів з питань охорони праці підтверджується записами у відповідних журналах обліку інструктажів і скріплюється підписами осіб, які проводили інструктажі та осіб, які отримали інструктажі.
Організаційні заходи перед початком, під час і після завершення роботи.
Працівники до початку роботи повинні перевірити візуально наявність і справність електрообладнання та його заземлення, а під час виконання роботи не залишати без нагляду обладнання, яке використовують. Після закінчення роботи необхідно прибрати робоче місце, відключити всі електроприлади від електромережі, перекрити крани водо- та газомережі.

\subsubsection{Санітарно-гігієнічні вимоги до умов праці}
Санітарно-гігієнічні вимоги до умов праці під час виконання роботи описують:
Нормативи з параметрів мікроклімату, освітлення приміщень, рівнів шуму, вібрації та електромагнітних випромінювань.
Мікроклімат виробничих приміщень характеризують температурою, вологістю та  швидкістю руху повітря, а також інтенсивністю радіації, переважно в інфрачервоній та ультрафіолетовій областях спектру електромагнітних випромінювань.
Параметри мікроклімату у приміщеннях повинні забезпечувати комфортне самопочуття організму. Тому у виробничих приміщеннях повинна бути надійна система кліматичного контролю. 
Параметри мікроклімату закритих приміщень нормують санітарні норми ДСН 3.3.6.042‑99. Оптимальні параметри мікроклімату закритих приміщень наведені в таблиці.
Оптимальні параметри мікроклімату закритих приміщень
Категорія
робіт	Температура, ºС	Відносна вологість повітря, %	Швидкість руху повітря, м/с			
холодний період року	теплий період року	холодний період року	теплий період року	холодний період року	теплий період року
Іа	22–24	23–25	60–40	60–40	0,1	0,1
Іб	21–23	22–24	60–40	60–40	0,1	0,2
Освітлення приміщень та робочих місць. Ще один важливий чинник, від якого залежать працездатність і здоров’я людини, – це освітлення. Світло регулює всі функції людського організму і впливає на психологічний стан і настрій, обмін речовин, гормональний фон і розумову активність. 
Найздоровіше освітлення забезпечує природне світло. Його ефективне використання можливе, якщо глибина приміщень не перевищує 6 м. Окрім того, хорошим вирішенням можуть будуть скляні перегородки, що забезпечують зорову і звукову ізоляцію, але в той же час не перешкоджають проникненню природного світла. 
Відносно вікон робоче місце повинно бути розміщено так, щоб природне світло було збоку, переважно з лівого та забезпечувати коефіцієнт природної освітленості не нижче 1,5 \%. Освітленість за штучного освітлення в площині робочої поверхні має становити 300–500 Лк. Якщо таких значень освітленості досягнути не можна, то допускається локальне освітлення, при цьому освітленість екрана не може перевищувати 300 Лк. Відношення яскравості робочих поверхонь не повинно бути більшим ніж 3:1, а яскравості робочих поверхонь і стін (іншого обладнання) - 5:1. Робоче місце, обладнане ПК повинно бути розташоване так, щоб уникнути попадання в очі прямого світла.
Щоб уникнути світлових відблисків від екрану та клавіатури необхідно використовувати комп’ютерне обладнання з матовою поверхнею. Для захисту очей від прямого сонячного світла чи джерел штучного освітлення необхідно застосовувати захисні козирки та жалюзі на вікнах. 
Вимоги до рівнів шуму та вібрації. Шум часто є причиною зниження рівня працездатності, підвищення рівня загальної та професійної захворюваності, частоти виробничих травм. Шум як стрес-чинник є загальнобіологічним подразником, який негативно впливає на всі органи і системи організму. У разі тривалого систематичного впливу шуму може виникнути патологія з переважним ураженням слуху, центральної нервової і серцево-судинної систем.
Гігієнічне нормування рівнів шуму. Допустимі рівні звукового тиску у октавних смугах частот, еквівалентні рівні звуку на робочих місцях встановлені санітарними нормами виробничого шуму, ультразвуку та інфразвуку ДСН 3.3.6.037-99, і для творчої та наукової роботи, навчання, не повинні перевищувати 50 дБА. 
Вібрація приводить тіло і його структурні частини в коливний рух. Розрізняють поперечні, поздовжні і крутильні коливання. За впливом на людину вібрації ділять на місцеві і загальні. Загальні вібрації викликають коливання тіла людини, місцеві – лише окремі частини тіла. Тривала дія вібрацій на організм людини призводить до порушень в центральній нервовій та серцево-судинній системах, погіршує загальний стан людини – появляються втома, головний біль тощо.
Колективні  та індивідуальні засоби і заходи захисту від шкідливого впливу виробничих чинників на здоров’я людини .
Облаштовуючи приміщення для роботи з ПК, потрібно передбачити припливно-витяжну вентиляцію або кондиціювання повітря. Надходження свіжого повітря регулюють, виходячи із таких умов (вказаний об’єм приміщення припадає на одне робоче місце з ПК):
\begin{itemize}
\item	якщо об’єм приміщення 20 м3, то потрібно подати не менш як 30 м3/год повітря;
\item	якщо об’єм приміщення у межах від 20 до 40 м3, то потрібно подати не менш як 20 м3/год повітря;
\item	якщо об’єм приміщення становить понад 40 м3, допускається природна вентиляція, у випадку, коли немає виділення шкідливих речовин.
\end{itemize}
Захист від шуму та вібрацій. Усунення шуму в приміщенні є однією з найскладніших проблем, оскільки джерела шуму різноманітні й потребують комплексу заходів технічного, організаційного і медичного характеру на всіх стадіях проектування, будівництва, експлуатації машин і устаткування. Відомі три головні напрямки зменшення впливу шуму на організм людини:
\begin{itemize}
\item	зменшення рівня шуму у джерелі виникнення, застосування раціональних конструкцій, нових матеріалів і технологічних процесів;
\item	звукоізоляція устаткування за допомогою глушників, резонаторів, кожухів, захисних конструкцій, оздоблення стін, стелі, підлоги тощо;
\item	використання засобів індивідуального захисту.
\end{itemize}
Заходи особистої гігієни на робочому місці (підтримання чистоти, миття лабораторного посуду, рук тощо).
Заходи особистої гігієни на робочому місці передбачають щоденне вологе прибирання, утримання у чистоті робочого місця, наявність на робочому місці тільки необхідних для роботи засобів. На робочому місці необхідно дотримуватись вимог правил внутрішнього розпорядку, зокрема, заборонено приймати їжу, пити, курити та ін.

\subsubsection{Заходи щодо безпеки під час виконання бакалаврської роботи}
1.	Заходи безпеки під час експлуатації персонального комп’ютера та периферійних пристроїв передбачають:
\begin{itemize}
\item	правильну організацію робочого місця та дотримання оптимальних режимів праці та відпочинку під час роботи з ПК;
\item	експлуатацію сертифікованого обладнання;
\item	дотримання заходів електробезпеки;
\item	забезпечення оптимальних параметрів мікроклімату;
\item	забезпечення  раціонального освітлення робочого місця;
\item	зниження рівня шуму та вібрації.
\end{itemize}
2.	Заходи безпеки під час експлуатації інших електричних приладів передбачають дотримання таких правил:
\begin{itemize}
\item	постійно стежити за справним станом електромережі, розподільних щитків, вимикачів, штепсельних розеток, лампових патронів, а також мережевих кабелів живлення, за допомогою яких електроприлади під’єднують до електромережі;
\item	постійно  стежити за справністю ізоляції електромережі та мережевих кабелів, не допускаючи їхньої експлуатації з пошкодженою ізоляцією;
\item	не тягнути за мережевий кабель, щоб витягти вилку з розетки;
\item	не закривати меблями, різноманітним інвентарем вимикачі, штепсельні розетки; 
\item	не підключати одночасно декілька потужних електропристроїв до однієї розетки, що може викликати надмірне нагрівання провідників, руйнування їхньої ізоляції, розплавлення і загоряння полімерних матеріалів;
\item	не залишати включені електроприлади без нагляду; 
\item	не допускати потрапляння всередину електроприладів крізь вентиляційні отвори  рідин або металевих предметів, а також не закривати їх та підтримувати в належній чистоті, щоб уникнути перегрівання та займання приладу;
\item	не ставити на електроприлади матеріали, які можуть під дією теплоти, що виділяється, загорітися (канцелярські товари, сувенірну продукцію).
\end{itemize}
\subsection{Безпека в надзвичайних ситуаціях}
\subsubsection{Протипожежні та противибухові заходи}
Пожежа - це неконтрольоване горіння, яке супроводжується виділенням тепла, світла, диму та інших продуктів. Горіння виникає за таких трьох умов: наявності окисника, наявності горючої речовини, наявності температури, за якої горюча речовина може самостійно горіти. Якщо немає хоча б однієї із цих умов, горіння стає неможливим. На цьому постулаті ґрунтується переважна більшість профілактичних заходів, спрямованих на відвернення пожеж.
У приміщенні в якому виконувалась бакалаврська робота не використовують пожежовибухонебезпечні речовини і матеріали. Найбільш ймовірним джерелом пожеж може бути несправність електрообладнання та загорянням горючих матеріалів, зокрема, канцелярського приладдя, паперу.
1.	Можливі причини виникнення пожежі та вибухів на робочому місці.
Головними причинами виникнення пожеж та вибухів є:
\begin{itemize}
\item	порушення пожежних норм і правил;
\item	порушення правил встановлення та експлуатації систем енергопостачання, опалення, вентиляції;
\item	порушення правил експлуатації електричного та газового обладнання;
\item	порушення правил зберігання пожежовибухонебез-печних матеріалів;
\item	використання відкритого вогню в заборонених місцях;
\item	погане знання персоналом протипожежних правил;
\item	необережна поведінка з вогнем.
\end{itemize}
2.	Заходи запобігання виникненню пожежі та вибуху, первинні засоби пожежогасіння.
Переважна більшість пожеж починається із невеличкого вогнища. Тому його своєчасну ліквідацію розглядаємо як профілактичний захід щодо недопущення його розширення до масштабів пожежі. Ліквідувати вогнище можна, усунувши одну із трьох умов виникнення горіння. Видалити горючу речовину із вогнища не завжди можна, а припинити доступ кисню до неї або/і понизити її температуру можна завжди, якщо своєчасно використати первинні засоби гасіння пожеж: воду, пісок або вогнегасники.
Вода – універсальний засіб для гасіння пожеж, оскільки її застосування завдяки випаровуванню дає змогу як понизити температуру горючої речовини, так і зменшити доступ кисню до неї. Проте нею не можна гасити електроустановки під напругою та легкозаймисті рідини. Для цього треба використовувати пісок, хоча він є менш ефективним.
Вогнегасники, залежно від природи вогнегасної речовини бувають різних типів. Найпоширеніші з них:
\begin{itemize}
\item	хімічно-пінні (ВХП–10);
\item	повітряно-пінні  (ВПП–5, ВПП–10);
\item	вуглекислотні (ВВ–2, ВВ–3, ВВ–5);
\item	порошкові (ВП–2–01, ВП–2Б, ВПУ–2, ВП–5–01, ВП–8Б).
\end{itemize}
Важливо пам’ятати, що хімічно-пінними та повітряно-пінними вогнегасниками не можна гасити електрообладнання під напругою. Потрібно своєчасно і вміло використовувати вогнегасники для локалізації невеликих ділянок горіння, оскільки час їхньої дії є досить малий, у найкращому випадку - до 60 с.
Також варто зазначити, що користуючись газовими приладами необхідно дотримуватись таких вимог:
\begin{itemize}
\item	забезпечити надійне вентилювання приміщення (відкрийте кватирки вікон, не закривайте отвори та решітки вентиляційних каналів, забезпечте постійне провітрювання приміщення);
\item	перевірити наявність тяги у димоході (за відсутності чи слабій тязі не користуйтесь газовими приладами – це смертельно небезпечно!);
\item	не встановлювати в приміщенні, де є газові прилади з відведенням продуктів згоряння у димові канали, витяжну механічну систему вентилювання;
\item	не користуватися газовими приладами за несправної автоматики безпеки;
\item	не залишати без нагляду газові прилади у робочому стані;
\item	не використовувати приміщення, де встановлено газові прилади, для сну та відпочинку;
\item	не використовувати газові плити для обігріву приміщення;
\item	перекривати крани після користування газовими приладами.
\end{itemize}

\subsubsection{Організація евакуації працівників}
Виконуючи бакалаврську роботу було проаналізовано схему шляхів евакуації, які забезпечують якнайшвидше і найбезпечніше виведення людей з небезпечних зон.
Проведення організованої евакуації з виробничих та інших приміщень і будівель, запобігання проявам паніки і недопущення загибелі людей забезпечують шляхом складання плану евакуації з розробленням схеми евакуаційних шляхів та виходів. На підприємстві має бути встановлено порядок оповіщення людей про пожежу, з яким необхідно ознайомити всіх працівників. 
Серед загальних вимог до евакуаційних шляхів та виходів необхідно відмітити, що ними можуть бути дверні отвори, якщо вони ведуть з приміщень: 
\begin{itemize}
\item	безпосередньо назовні; 
\item	на сходовий майданчик з виходом назовні безпосередньо або через вестибюль; 
\item	у прохід або коридор з безпосереднім виходом назовні або на сходовий майданчик; 
\item	у сусідні приміщення того ж поверху, що не містять виробництв, які належать за вибухопожежною та пожежною небезпекою до категорій А, Б і В та мають безпосередній вихід назовні або на сходовий майданчик. 
\end{itemize}
Для безпечної евакуації шляхи та виходи мають відповідати таким вимогам: 
\begin{itemize}
\item	евакуаційні шляхи і виходи повинні утримуватися вільними, не захаращуватися та у разі потреби забезпечувати евакуацію всіх людей, які перебувають у приміщеннях;
\item	кількість та розміри евакуаційних виходів, їхні конструктивні рішення, умови освітленості, забезпечення незадимленості, протяжність шляхів евакуації, їхнє оздоблення повинні відповідати протипожежним вимогам будівельних норм;
\item	у приміщенні, яке має один евакуаційний вихід, дозволяється одночасно розміщувати не більше 50 осіб, а у разі перебування в ньому понад 50 осіб повинно бути щонайменше два виходи, які відповідають вимогам будівельних норм;
\item	двері на шляхах евакуації повинні відчинятися в напрямку виходу з будівель (приміщень) і замикатися лише на внутрішні запори, які легко відмикаються.
\end{itemize}

\clearpage
 
\section{Висновки}

Внаслідок виконання цієї магістерської роботи було створено декілька інструментів для спрощення досліджень алгоритмів моделювання процесів на графах. З використанням їх було досліджено та проведено спроби адаптації двох алгоритмів моделювання логічних схем -- одно- та двопрохідні алгоритми базовані на аналізі через події.

Перед апробацією вищезазначених методів було проведено роботу по більш глибокому вивченню предметної області, розглянуто класифікацію алгоритмів моделювання та, згідно з нею, описано наявний в BUMMEL алгоритм. Також було проведено роботу по опису наявної архітектури моделі програми, що допоможе швидшій інтеграції нових учасників проекту.

В якості одного з інструментів сприяння дослідженню було розроблено динамічне підвантаження наявних схем моделювання в меню і можливість одночасної роботи з різними схемами для порівняльного аналізу, наприклад.

Також, проведена робота привела до наступних висновків та ідей. Синтаксис мови Java є доволі громіздкий, а це ускладнює розробку чи апробацію алгоритмів та архітектурних елементів. На даному проекті, але в іншій роботі проводиться експеримент з переведенням частини програми на функціональну мову програмування Scala. Дане дослідження підтверджує актуальність таких експериментів і пошуку зручнішого синтаксису. Крім того, Java - компілятивна мова програмування і в межах даного проекту, який є доволі великий, багато часу іде на сам процес компіляції. Тому постає питання спроби використання інтерпретативних мов, де б можна було зразу ж подивитись результати експерименту. Це дуже зручно, особливо з точки зору досліджень.

Тестування та апробація алгоритмів моделювання логічного рівня показала, що поєднати ідею універсальної бази для моделювання процесів на мережах, графах і т.д. доволі складно. В кожному з випадків процес затримувався в певному місці, без видимого шляху доведення експерименту до працюючого результату. Для вирішення проблеми моделювання потрібно продовжити дослідження та підключити до співпраці людей з глибоким розумінням предметної області, так як інколи потрібно одночасно і широко та здалеку оглядати проблему, і пам’ятати про дрібниці в глибині моделювання.

Приємно зазначити, що за рік цієї роботи BUMMEL було представлено на двох конференціях: один
%\cite{}
, два \cite{thesis-foss2013}. Можливо, це допоможе залученню до проекту сторонніх розробників вільного програмного забезпечення чи наукових лиць, зацікавлених даною тематикою.

Частково це вже відбулось, так як на конференціях поступало декілька пропозицій, які цікаві на сьогоднішній день і можуть бути реалізовані з деякою зміною даного функціоналу.

Також, одною з цікавих напрямків розвитку, думка та ідея про який з’явилась внаслідок проведених досліджень та співпраці з Тимчуком Ю.А. та ознайомленням з його магістерською роботою
%\cite{thesis-Tymchuk}
, є переведення BUMMEL з електронної лабораторії в платформу для досліджень моделювання різних процесів, або побудова такої на базі цієї програми.

Приносить радість те, що над проектом працює жвава команда розробників, які привносять кожен щось своє і нове в даний продукт, діляться цікавими конструктивними ідеями та демонструють практичні рішення до різних проблем, що постають на шляху розвитку електронної лабораторії BUMMEL.

\subsection{?БЖД висновки}
Розділ закінчують висновками, в яких студенти встановлюють відповідність умов праці санітарно-гігієнічним та ергономічним нормативам і подають пропозиції для покращення умов праці на робочому місці. 

\clearpage

\addcontentsline{toc}{section}{Список використаних джерел}
\begin{thebibliography}{9}

%  \bibitem{alias}<АВТОР> \emph{<КНИГА>}
%     - <МІСЦЕ ВИДАННЯ> : <ВИДАВНИЦТВО>, <РІК>. - <К-КІСТЬ СТОРІНОК> с.

%  \bibitem{alias2}<АВТОР> \emph{<НАЗВА>} [Електронний ресурс],
%    <РІК>. Режим доступу:
%    \url{https://github.com/Uko/thesis-template}

  \bibitem{nb-6.9-dev-guide}Jürgen Petri \emph{NetBeans Platform 6.9 Developer's Guide}
     - Birmingham : Packt Publishing, 2010. - 275 с.

  \bibitem{nb-7-dev-guide}Heiko Böck \emph{The Definitive Guide to NetBeans Platform 7}
     - New York : APRESS, 2012. - 562 с.

  \bibitem{nb-7-cook}Rhawi Dantas \emph{NetBeans IDE 7 Cookbook}
     - Birmingham : Packt Publishing, 2011. - 308 с.

  \bibitem{thesis-foss2013}FOSS 2013 Conference in Lviv

  \bibitem{n4}

  \bibitem{n5}

  \bibitem{n6}

  \bibitem{n7}

  \bibitem{n8}

  \bibitem{n9}

  \bibitem{n10}

  \bibitem{n11}

  \bibitem{n12}

  \bibitem{n13}

  \bibitem{n14}

  \bibitem{n15}

  \bibitem{n16}

  \bibitem{n17}

  \bibitem{n18}

\end{thebibliography}

\end{document}
